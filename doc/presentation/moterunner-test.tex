\section{Testing Mote Runner}
\begin{frame}[fragile]
  \frametitle{Problems}
  \begin{itemize}
    \item MR v.13 offers:
    \begin{itemize}
    	\item Radio interface IEEE 802.15.4 compliant
    	\item Hopi
    	\item A simulation environment IRIS friendly
    	\item Many nice features (Debugger, Logger and so on)
    \end{itemize}
  \end{itemize}
\end{frame}

\begin{frame}[fragile]
  \frametitle{Programming the Radio}
  \begin{itemize}
    \item com.ibm.saguaro.system.Radio
    \begin{itemize}
    	\item This is a generic class in the IBM saguaro system to use the device radio
    	\item It offers a low level API with the following functionality:
    	\begin{itemize}
	  \item open: opens the radio, once opened no other assembly can use it
	  \item close: releases the radio so that others can use it
	  \item setter and getters for channel and network parameters (addresses, panid...)
	  \item startReceive: listens the channel (in one of the many receiption mode)
	  \item transmit: begin to transmit a pdu
    	\end{itemize}
    \end{itemize}
  \end{itemize}
\end{frame}

\begin{frame}[fragile]
  \frametitle{Transmission \& Reception}
  \begin{itemize}
    \item These operations require much attention:
    \begin{itemize}
    	\item The radio permits to transmit every type of pdu, but it's possible to receive only 802.15.4 well formed packets
    	\item It's also possible to receive in promiscuous mode to sniff for every packet
    	\item However it's suitable to adopt IEEE 802.15.4 packets to reduce interferences in reception
    	\item 
    \end{itemize}
    
  \end{itemize}
\end{frame}