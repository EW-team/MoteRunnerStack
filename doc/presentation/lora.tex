\section{LoRaWAN}
  
\begin{frame}[fragile]
  \frametitle{A deeper look at LoRa}
  \begin{itemize}
      \item LoRaWAN is a Low Power Wide Area Network (LPWAN) specification intended for wireless battery operated Things in regional, national or global network. 
      \item LoRaWAN target key requirements of internet of things:
      \begin{itemize}
      	\item secure bi-directional communication
      	\item mobility - Non molto materiale se non quello su ADR
      	\item localization services - INDAGARE(Beaconing)
      \end{itemize}
  \end{itemize}
\end{frame}

\begin{frame}[fragile]
  \frametitle{LoRa devices sync}
  For time synchronization gateways periodically broadcast so-called beacons. Each beacon
  minimally contains:
  \begin{itemize}
      \item available channels (ChMask)
      \item current GPS time (Time)
      \end{itemize}
      The broadcasting of beacons (in implicit mode) is done time-synchronously (BEACON\_INTERVAL) by all gateways of a network with no interference.
      \begin{figure}
  \centering
  \includegraphics[width=\textwidth]{img/lora_beaconing.png}
  \end{figure}
\end{frame}

\begin{frame}[fragile]
  \frametitle{LoRa MAC Payload Frame}
  \begin{columns}
    \begin{column}{.48\linewidth}
    LoRa MAC message types:
       \begin{itemize}
        \item join request
        \item join accept 
        \item unconfirmed data messages
        \item confirmed data messages
       \end{itemize}
  \end{column}
   \begin{column}{.48\linewidth}
 \begin{figure}
  \centering
  \includegraphics[width=\textwidth]{img/mac_frame.png}
  \end{figure}
  \end{column}
  \end{columns}
\end{frame}

\begin{frame}[fragile]
  \frametitle{LoRa end-devices}
  \begin{itemize}
   \item  Release v1.0 allows at MAC and Application layers, Bi-directional communications:
     \begin{itemize}
       \item Class A: after send operation two tiny time windows are opened in order to allow reception
       \item Class B: send and receive operations may be scheduled based on the time information contained in the beacons
     \end{itemize}
    \item  Last Release = Release v1.0 plus:
     \begin{itemize}
       \item Class C: nearly continuously open receive windows, only closed when transmitting.
     \end{itemize}
  \end{itemize}
  \begin{figure}
  \centering
  \includegraphics[width=0.4\textwidth]{img/lora_rx_windows.png}
  \end{figure}

\end{frame}


\begin{frame}[fragile]
  \frametitle{LoRa secure communications - 1}
  In order to partecipate in a LoRa network an end device first has to be personalized and the activated.
  Activation of an end device can be achieved in two ways:
  \begin{itemize}
    \item OTAA (over-the-air activation) when an end device is deployed or reset;
    \item APB (activation by personalization) one-step personalization and activation.
  \end{itemize}
\end{frame}

\begin{frame}[fragile]
  \frametitle{LoRa secure communications - 2}
  During activation the end device holds the following informations:
  \begin{itemize}
      \item \textbf{DevAddr:} device ID of 32 bits that uniquely identifies the end device.
      \item \textbf{AppEUI:} globally unique application ID that uniquely identifies the application provider of the end device.
      \item \textbf{NwkSKey:} device-specific network session key, ensures data integrity and os used to encrypt/decrypt MAC data messages payload
      \item \textbf{AppSKey:} device-specific application session key, used to encrypt and decrypt the payload field of application-specific data messages.
  \end{itemize}
\end{frame}


\begin{frame}[fragile]
  \frametitle{LoRa secure communications - OTAA}
 \begin{figure}
  \centering
  \includegraphics[width=0.9\textwidth]{img/lora_secure.png}
  \end{figure}
\end{frame}

\begin{frame}[fragile]
  \frametitle{LoRa mobility}
  LoRa network data rates are:
 \begin{itemize}
  \item Network controlled for fixed devices by means of using ADR bit in the PHY payload of data messages:
  \begin{itemize}
    \item If is set, the network will control the data rate of the end device through the appropriate MAC commands
    \item If is cleared, the network will not attempt to control the data rate of the end device independently of the received signal quality
  \end{itemize}
  \item default for mobile end-devices.
\begin{figure}
  \centering
  \includegraphics[width=0.9\textwidth]{img/PHYpayload.png}
  \end{figure}
 \end{itemize}

\end{frame}

\begin{frame}[fragile]
  \frametitle{LoRa mobility 2}
 \begin{itemize}
  \item \hypertarget{refthis}{}
  \item However, whenever possible, the ADR scheme should
be enabled to increase the battery life of the end device and maximize the network capacity
 \end{itemize}

\end{frame}